\documentclass{article}
\usepackage{graphicx}
\usepackage[utf8]{inputenc}
\usepackage{array}
\usepackage{tabularx}
\usepackage[includeheadfoot,margin=2cm]{geometry}
\usepackage[font=small,labelfont=bf,tableposition=bottom]{caption}
\usepackage{booktabs}
\usepackage[francais]{babel}

\begin{document}

\title{Titre}
\author{Antoine PUISSANT}

\maketitle

\begin{abstract}
The abstract text goes here.
\end{abstract}
\newpage
\tableofcontents
\newpage
\section{Exercice 1}
\subsection{Question 1}
Somme des carr\'es de regression par la somme des carr\'es r\'esiduels.\newline
\begin{equation}
  \label{simple_equation}
  R^{2} = \frac{SC_{reg}}{SC_{tot}}
\end{equation}
\begin{equation}
  \label{simple_equation}
  R^{2} = \frac{980.64}{980.64 + 440.03} = 0.69 = 69\%
\end{equation}

\begin{equation}
  \label{simple_equation}
  R^{2} = \frac{643.57}{643.57 + 777.10} = 0.45 = 69\%
\end{equation}

\subsection{Question 2}
Il faut renmplir le tableau avec le carré moyen de la variance.
Somme des carrés résiduels divisé par le degré de liberté.
\begin{equation}
  \label{simple_equation}
  \frac{SC_{res}}{n - p}
\end{equation}
\begin{table}[!ht]
  \centering
  \footnotesize
  \begin{tabular}{lcc}\toprule
    & Carré moyen résiduel & \'Ecart-type des résidus\\\midrule
    Régression due à \(X_{1}\) & \(\frac{440.03}{10} = 44.003\) & \(\sqrt{44.003} = 6.6334\)\\
    Régression due à \(X_{2}\) & \(\frac{777.10}{10} = 77.710\) & \(\sqrt{77.710} = 8.815\)\\
    Régression due à \(X_{1}, X_{2}\) & \(\frac{215.81}{9} = 23.979\) & \(\sqrt{23.979} = 4.896223\)\\\bottomrule
  \end{tabular}
  \caption{Question 2}\label{tab:table}
\end{table}
\subsection{Question 3}
\begin{table}[!ht]
  \centering
  \footnotesize
  \begin{tabular}{lcccc}\toprule
    Source de variation & Somme des carrés & ddl & Carrés moyens & \(F_{obs}\)\\\midrule
    Régression due à \(X_{1}, X_{2}\) & 1204.86 & \(p - 1 = 3 - 1 = 2\) & 602.43 & 25.12\\
    Résiduelle & 215.81 & \(p - 1 = 10 - 1 = 9\) & 23.98\\
    Totale & 1420.67 & \(p - 1 = 12 - 1 = 11\)\\\bottomrule
  \end{tabular}
  \caption{Question 3}\label{tab:table}
\end{table}
\subsection{Question 4}
On va selectionner une test de Fisher car on veut tester les coéfficents de la régression. 
\subsection{Question 5}
\subsection{Question 6}
\subsection{Question 7}
\subsection{Question 8}
Pour  savoir si la contribution marginale de la variable "densité du matériau" est significative lorsqu'elle est introduite à la suite de la variable "épaisseur du matériau" pour un seuil de signification \(\alpha = 5\%\) on va alors réaliser deux tests :
\begin{itemize}
	\item Le test de Student\\
	On teste sur le \(\widehat{\beta_{2}}\)\( = 11.072\). Soit :\\
	\(\frac{11.072}{3.621} = 3.05\)\\
	On calcule ensuite le quantile du test de Student :\\
	\(qt(0.975, 9) = 2.26\) On a $\alpha = 5\%$ donc pour qt, on a \(1 - (\frac{\alpha}{2}) = 0.975\). 9 correspond au 12 variables moins les 3 utilisées.
	\item Le test de Fisher partiel
\end{itemize}
\subsection{Question 9}
Il faut modéliser avec les variables \(X_{1}/X_{2}\) avec la commande 'predict'.
\begin{table}[!ht]
  \centering
  \footnotesize
  \begin{tabular}{cccc}\toprule
    \'Epaisseur $X_{1}$ & Densité $X_{2}$ & Estimation de la résistance moyenne & \'Ecart-type de l'estimation\\\midrule
    4 & 3.8 & 31.61175 & 2.10\\
    3 & 3.4 & 22.27821 & 1.43\\
    4 & 2.9 & 21.64689 & 2.57\\\bottomrule
  \end{tabular}
  \caption{Question 9}\label{tab:table}
\end{table}\newline
Sur R, le \textit{residual standard error} = $\frac{\sqrt{SC_{Residual}}}{ddl}$
\subsection{Question 10}
Pour trouver l'intervalle de confiance, on tape dans R :\newline
\textit{$predict(modele1, data.frame(Ep_mat = 4, Dens = 3.8), interval="confidence")$}\newline
On obtient alors l'intervalle de confiance suivant : $[26.86038, 36.36311]$
\subsection{Question 11}
Pour trouver la marge d'erreur, on fait : \(36.36311 - 26.86038 = 9.5\%\)
\subsection{Question 12}
Pour trouver l'intervalle de prédiction, on tape dans R :\newline
\textit{$predict(modele1, data.frame(Ep_mat = 4, Dens = 3.8), interval="prediction")$}\newline
On obtient alors l'intervalle suivant : \([19.55839, 43.6651]\)
\newpage
\section{Section 2}
Here is the text of your introduction.

\begin{equation}
    \label{simple_equation}
    \alpha = \frac{\sqrt{ \beta }}{\gamma}
\end{equation}

\subsection{Subsection}
Write your subsection text here.
\newpage
\section{Conclusion}
Write your conclusion here.

\end{document}
